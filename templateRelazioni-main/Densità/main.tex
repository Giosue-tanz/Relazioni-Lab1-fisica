\documentclass{article}
\usepackage[a4paper, tmargin=2cm,rmargin=1.5in,lmargin=1.5in,margin=0.85in,bmargin=2cm,footskip=.2in]{geometry}
\usepackage{bookmark}
\usepackage{booktabs}
\usepackage{listings}
\usepackage{amsmath,amsfonts,amsthm,amssymb,mathtools}
\usepackage[italian]{babel}
\usepackage{graphicx}
\graphicspath{{img/}}

\usepackage{multirow}
\usepackage{subfloat}
\usepackage{wrapfig}
\usepackage{siunitx}
\usepackage{float}
\usepackage{caption}
\usepackage{setspace}
\usepackage{ragged2e}
\usepackage{longtable}
\usepackage[T1]{fontenc}
\usepackage{hyperref}

\usepackage{xcolor}
\usepackage{titling}
\renewcommand\maketitlehooka{\null\mbox{}\vfill}
\renewcommand\maketitlehookd{\vfill\null}
\definecolor{codegreen}{rgb}{0,0.6,0}
\definecolor{codegray}{rgb}{0.5,0.5,0.5}
\definecolor{codepurple}{rgb}{0.58,0,0.82}
\definecolor{backcolour}{rgb}{0.95,0.95,0.92}
\usepackage{subcaption}

% Label format
\DeclareCaptionLabelFormat{custom}
{%
      \textsc{#1 \textbf{(#2)}}
}
% Separator style
\DeclareCaptionLabelSeparator{custom}{--}
% Caption format    
\DeclareCaptionFormat{custom}
{%
    #1#2 \small #3
}
\captionsetup{
	format=custom,
	labelformat=custom,
	labelsep=custom
}

\lstdefinestyle{code}{
    backgroundcolor=\color{backcolour},   
    commentstyle=\color{codegreen},
    keywordstyle=\color{magenta},
    numberstyle=\tiny\color{codegray},
    stringstyle=\color{codepurple},
    basicstyle=\ttfamily\footnotesize,
    breakatwhitespace=false,         
    breaklines=true,                 
    captionpos=b,                    
    keepspaces=true,                 
    numbers=left,                    
    numbersep=5pt,                  
    showspaces=false,                
    showstringspaces=false,
    showtabs=false,                  
    tabsize=2
}

\usepackage{hyperref}
\hypersetup{
    colorlinks=true,
    linkcolor=black,
    pdfborder={0 0 0},
	urlcolor=black,
}

\setlength{\parindent}{0pt}

%configurazione disegni
\usepackage{tikz}

% Configurazione dello stile per il codice Python
\usepackage{listings}
\usepackage{xcolor}

% Definizione dei colori
\definecolor{codegreen}{rgb}{0,0.6,0}
\definecolor{codegray}{rgb}{0.5,0.5,0.5}
\definecolor{codepurple}{rgb}{0.58,0,0.82}
\definecolor{backcolour}{rgb}{0.95,0.95,0.92}
\lstdefinestyle{mystyle}{
    backgroundcolor=\color{backcolour},   
    commentstyle=\color{codegreen},
    keywordstyle=\color{magenta},
    numberstyle=\tiny\color{codegray},
    stringstyle=\color{codepurple},
    basicstyle=\ttfamily\footnotesize,
    breakatwhitespace=false,         
    breaklines=true,                 
    captionpos=b,                    
    keepspaces=true,                 
    numbers=left,                    
    numbersep=5pt,                  
    showspaces=false,                
    showstringspaces=false,
    showtabs=false,                  
    tabsize=2,
    language=Python
}

\lstset{style=mystyle}

%new
\usepackage{cleveref}

% Personalizzazione del formato dei riferimenti alle tabelle
\crefname{table}{Tabella}{Tabelle}
\Crefname{table}{Tabella}{Tabelle}
\title{Relazione densità}
\author{Giosuè Aiello}
\date{\today}
\begin{document}
	\maketitle
	\section{Scopo}
	Scrivere scopo una volta all'esame
	\section{Premesse teoriche}
	La densità ($\rho$) viene definito come il rapporto fra la massa e il volume:
	\begin{equation}
		\rho = \frac{m}{V}
	\end{equation}
	dove $m$ rappresenta la massa del corpo e $V$ il volume dell'oggetto e della sostanza considerata. La densità è una proprietà intrinseca che caratterizza ogni materiale e può essere utilizzata per identificare o distinguere fra diversi materiali. \\
	Nel caso di sfere di differente raggio $r$ e con la medesima densità $\rho$ possiamo affermare che
	\begin{equation}
	m = \frac{4}{3} \pi \rho r^3 = kr^3
	\end{equation}
	e rappresentandolo in scala bilogaritmica, questa relazione appare come una retta con il coefficiente angolare della retta corrispondente al valore atteso 3.
\section{Strumenti e materiali}
	\textbf{Strumenti}:	
	\begin{itemize}
		\item Calibro ventesimale (risoluzione $0.05 \, \si{\milli\meter}$)
		\item Calibro Palmer (risoluzione $0.01 \si{\milli\meter}$)
		\item Bilancia di precisione (risoluzione $1 \si{\milli\gram}$)
	\end{itemize}
	\textbf{Materiali}:
	\begin{itemize}
		\item una serie di solidi
	\end{itemize}
\section{Descrizione delle misure}
Per effettuare questa esperienza abbiamo prima verificato la corretta taratura della bilancia che abbiamo utilizzato per svolgere l'esperienza. \\
Successivamente, abbiamo preso i vari solidi e abbiamo misurato la loro massa e tutti i parametri necessari per determinare il loro volume: (scrivere qua i parametri per tipologia di solido)
\section{Analisi delle misure}
In questa esperienza, gli errori sulle ascisse (ovvero gli errori sul volume) non sono trascurabili rispetto agli errori sulle ordinate (ovvero sulle masse), ovvero
$$
	\sigma_{m_i} << \frac{df}{dV_i} \sigma_{V_i}
$$
dove nel nostro caso $f(V_i) = \rho V_i$ e la $i$ indica l'$i$-esimo solido considerato. \\
Dunque, per semplificare la trattazione, ho deciso di effettuare il grafico massa-volume, dunque gli errori sulle ascisse (che adesso rappresenta la massa) sono adesso trascurabili e abbiamo deciso di effettuare il fit con il seguente modello teorico:
$$
	V_i = \frac{m_i}{\rho}
$$
dunque i dati si dovrebbero ancora disporre su una retta con coefficiente angolare pari a $\frac{1}{\rho}$
Per quanto riguarda l'incertezza assegnata al volume $V_i$, ho propagato l'errore nella seguente maniera: (scrivere in base ai solidi)
\end{document}