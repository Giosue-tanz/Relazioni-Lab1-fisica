\documentclass{article}
\usepackage[a4paper, tmargin=2cm,rmargin=1.5in,lmargin=1.5in,margin=0.85in,bmargin=2cm,footskip=.2in]{geometry}
\usepackage{bookmark}
\usepackage{booktabs}
\usepackage{listings}
\usepackage{amsmath,amsfonts,amsthm,amssymb,mathtools}
\usepackage[italian]{babel}
\usepackage{graphicx}
\graphicspath{{img/}}

\usepackage{multirow}
\usepackage{subfloat}
\usepackage{wrapfig}
\usepackage{siunitx}
\usepackage{float}
\usepackage{caption}
\usepackage{setspace}
\usepackage{ragged2e}
\usepackage{longtable}
\usepackage[T1]{fontenc}
\usepackage{hyperref}

\usepackage{xcolor}
\usepackage{titling}
\renewcommand\maketitlehooka{\null\mbox{}\vfill}
\renewcommand\maketitlehookd{\vfill\null}
\definecolor{codegreen}{rgb}{0,0.6,0}
\definecolor{codegray}{rgb}{0.5,0.5,0.5}
\definecolor{codepurple}{rgb}{0.58,0,0.82}
\definecolor{backcolour}{rgb}{0.95,0.95,0.92}
\usepackage{subcaption}

% Label format
\DeclareCaptionLabelFormat{custom}
{%
      \textsc{#1 \textbf{(#2)}}
}
% Separator style
\DeclareCaptionLabelSeparator{custom}{--}
% Caption format    
\DeclareCaptionFormat{custom}
{%
    #1#2 \small #3
}
\captionsetup{
	format=custom,
	labelformat=custom,
	labelsep=custom
}

\lstdefinestyle{code}{
    backgroundcolor=\color{backcolour},   
    commentstyle=\color{codegreen},
    keywordstyle=\color{magenta},
    numberstyle=\tiny\color{codegray},
    stringstyle=\color{codepurple},
    basicstyle=\ttfamily\footnotesize,
    breakatwhitespace=false,         
    breaklines=true,                 
    captionpos=b,                    
    keepspaces=true,                 
    numbers=left,                    
    numbersep=5pt,                  
    showspaces=false,                
    showstringspaces=false,
    showtabs=false,                  
    tabsize=2
}

\usepackage{hyperref}
\hypersetup{
    colorlinks=true,
    linkcolor=black,
    pdfborder={0 0 0},
	urlcolor=black,
}

\setlength{\parindent}{0pt}

%configurazione disegni
\usepackage{tikz}

% Configurazione dello stile per il codice Python
\usepackage{listings}
\usepackage{xcolor}

% Definizione dei colori
\definecolor{codegreen}{rgb}{0,0.6,0}
\definecolor{codegray}{rgb}{0.5,0.5,0.5}
\definecolor{codepurple}{rgb}{0.58,0,0.82}
\definecolor{backcolour}{rgb}{0.95,0.95,0.92}
\lstdefinestyle{mystyle}{
    backgroundcolor=\color{backcolour},   
    commentstyle=\color{codegreen},
    keywordstyle=\color{magenta},
    numberstyle=\tiny\color{codegray},
    stringstyle=\color{codepurple},
    basicstyle=\ttfamily\footnotesize,
    breakatwhitespace=false,         
    breaklines=true,                 
    captionpos=b,                    
    keepspaces=true,                 
    numbers=left,                    
    numbersep=5pt,                  
    showspaces=false,                
    showstringspaces=false,
    showtabs=false,                  
    tabsize=2,
    language=Python
}

\lstset{style=mystyle}

%new
\usepackage{cleveref}

% Personalizzazione del formato dei riferimenti alle tabelle
\crefname{table}{Tabella}{Tabelle}
\Crefname{table}{Tabella}{Tabelle}
\title{Relazione calcolo di $g$}
\author{Giosuè Aiello}
\date{\today}
\begin{document}
	\maketitle
	\section{Scopo dell'esperienza}
	Calcolo dell'accelerazione di gravità $g$ tramite una molla
	\section{Premesse teoriche}
	Una molla è un corpo in grado di allungarsi e accorciarsi se gli viene applicata una forza e in seguito di ritornare alla propria forma naturale. Tramite la legge di Hooke sappiamo che essa reagisce esercitando una forza che reagisce alle sollecitazione subita longitudinalmente, in trazione o in compressione, lungo un asse $\hat{x}$
	\begin{equation}
	\vec{F_e} = - k \Delta l \hat{x}
	\end{equation}
	quindi si osserva che essa è direttamente proporzionale all'allungamento o alla compressione $\Delta l$ (che dimensionalmente ha come unità di misura quella di una lunghezza $[L]$) della molla dovuto alla sollecitazione e la costante di questa proporzionalità $k$ si chiama \emph{costante elastica della molla} (che invece dimensionalmente ha l'unità di misura di una forza diviso una lunghezza, dunque $\frac{[L][T]^{-2}[M]}{[L]} = [M][T]^{-2}$). \\
	La forza di gravità esercitata dalla Terra su un corpo che si trova sulla sua superficie è pari a
	$\vec{F} = m\vec{g}$
	dove $\vec{g}$ è l'accelerazione di gravità sulla superficie terrestre. E' possibile stimare il valore di $g=|\vec{g}|$ misurando l'allungamento della molla dovuto all'azione di una massa appesa ad una estremità della molla, pertanto:
	\begin{equation}
		m|\vec{g}| = k \Delta l \implies |\vec{g}| = \frac{k}{m} \Delta l
		\label{secondo_modello}
	\end{equation}
	Tramite una stima della lunghezze $\Delta l$, $k$ e conoscendo la  massa $m$ appesa possiamo stimare g. Per fare ciò, ci avvarremo della formula del periodo $T$ delle oscillazioni compiute dalla molla con la massa appesa
	\begin{equation}
		T = 2 \pi \sqrt{\frac{m}{k}}
	\end{equation}
	Da cui risulta che 
	\begin{equation}
	T^2 = \frac{4\pi^2}{k} m
	\label{primo_modello}
	\end{equation}
	dunque il grafico $T^2 - m$ sarà una retta con coefficiente angolare pari a $\frac{4\pi^2}{k}$ da cui è possibile ottenere una stima di $k$.
	\section{Apparato sperimentale}
	\textbf{Strumenti}:
	\begin{itemize}
		\item metro a nastro con sensibilità pari a $0.1 \si{\centi\meter}$
		\item cronometro con risoluzione pari a $0.001 \si{\second}$
	\end{itemize}		
	\textbf{Materiali}:
	\begin{itemize}	
		\item pesini di metallo di masse differenti
		\item piattino di metallo
		\item molla
	\end{itemize}
	\section{Descrizione delle misure}
	Prima di procedere abbiamo misurato la massa del piattino, che risultava essere pari a (inserire). \\ Successivamente abbiamo effettuato le misure necessarie per stimare la costante elastica della molla $k$, misurando le oscillazioni della molla soggetta alla forza peso esercitata dalla massa del piattino e di un pesino posto sopra utilizzando varie masse, che riportiamo in una tabella qua accanto:
	
	\begin{wraptable}{r}{0.5\textwidth}
		\centering
		\begin{tabular}{c c}
		\toprule
		$m_i$ & $T_i$ \\
		$\pm \si{\kilo\gram}$ & $\pm \si{\second}$ \\
		\toprule
		

		\bottomrule		
		\end{tabular}
	\end{wraptable}
	\noindent Per ogni massa abbiamo misurato $10$ volte il periodo di oscillazione e, successivamente, come risultato della nostra misura abbiamo preso per il pesino $i$-esimo:
	$$
		\hat{T_i} = T_{m_i} \pm \sqrt{\frac{1}{10 \cdot 9} \sum_{j=1}^9 (T_j - T_{m_i})^2}
	$$
	dove $T_{m_i}$ rappresenta la media aritmetica di tutti i periodi di oscillazione che abbiamo misurato per l'$i$-esimo pesino e $T_j$ rappresenta invece il $j$-esimo valore misurato per l'$i$-esimo pesino. \\
	Successivamente, abbiamo effettuato misurato gli allungamenti della molla con masse appese diverse che riportiamo nella tabella qua accanto
	\section{Analisi delle misure}
	Per effettuare la stima della costante elastica $k$ della molla ho deciso di farmelo stimare tramite la procedura del fit dei minimi quadrati tenendo $k$ come parametro libero e utilizzando come modello la (\ref{primo_modello}). Si riporta il grafico qua sotto:
	DA INSERIRE
	
\noindent Per determinare $g$ abbiamo invece deciso di effettuare sempre un'analisi tramite fit dei minimi quadratici tenendo $g$ come parametro libero, riportando i valori di $g$ che è possibile ottenere con la (\ref{secondo_modello}) in un grafico $g_i - m_i$ e fittando con una retta del tipo $y=c$, che riportiamo qua sotto:
DA INSERIRE
\end{document}
